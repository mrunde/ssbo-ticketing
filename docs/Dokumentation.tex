\documentclass[12pt,a4paper,final,twoside,onecolumn,titlepage]{article}
\usepackage[utf8]{inputenc}
\usepackage[german]{babel}
\usepackage{graphicx}
\usepackage[left=2cm,right=2cm,top=2cm,bottom=2cm]{geometry}

\author{Marius Runde}
\title{SSBO-Ticketing}

\begin{document}
\maketitle
\tableofcontents

\section{Einrichtung der Datenbank}

Um die Datenbank einzurichten, muss folgender Code im SQL-Editor eingegeben werden:\\\\
\texttt{\hspace*{10mm}CREATE DATABASE ssbo\_ticketing;}\\\\
Danach muss die notwendige Tabelle erstellt werden:\\\\
\texttt{\hspace*{10mm}CREATE TABLE IF NOT EXISTS tickets (\\
	\hspace*{20mm}code		BIGINT UNSIGNED NOT NULL AUTO\_INCREMENT,\\
	\hspace*{20mm}firstname	VARCHAR(50) NOT NULL,\\
	\hspace*{20mm}lastname	VARCHAR(50) NOT NULL,\\
	\hspace*{20mm}street		VARCHAR(50) NOT NULL,\\
	\hspace*{20mm}housenumber	VARCHAR(5) NOT NULL,\\
	\hspace*{20mm}postalcode	INT(5) UNSIGNED NOT NULL,\\
	\hspace*{20mm}town		VARCHAR(50) NOT NULL,\\
	\hspace*{20mm}date		TIMESTAMP NOT NULL,\\
	\hspace*{20mm}used		BOOLEAN NOT NULL DEFAULT false,\\
	\hspace*{20mm}PRIMARY KEY (code)\\
\hspace*{10mm});}

\section{Einbinden in die Website}

Um das SSBO-Ticketing-System in die Website einzubinden, müssen die Dateien \texttt{index.php}, \texttt{control.php}, \texttt{db\_info.php}, \texttt{new.php} und \texttt{style.css} sowie die Verzeichnisse \texttt{img}, \texttt{js} und \texttt{lib} in das Verzeichnis im Webspace kopiert werden, über welchen Pfad später die Tickets gebucht werden sollen können.

In der \texttt{db\_info.php} müssen die Zugangsdaten für die Datenbank eingetragen werden.

\section{Hinzufügen eines neuen Tickets zur Datenbank}

Neue Tickets lassen sich mit der Datei \texttt{new.php} in der Datenbank eintragen.

\end{document}